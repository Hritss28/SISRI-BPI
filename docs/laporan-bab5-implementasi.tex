% ============================================
% BAB 5 - IMPLEMENTASI SISTEM
% SISRI-BPI (Sistem Informasi Skripsi)
% Universitas Trunojoyo Madura
% ============================================

\chapter{IMPLEMENTASI SISTEM}

\section{Lingkungan Implementasi}

\subsection{Spesifikasi Perangkat Keras}
Implementasi sistem SISRI-BPI dilakukan pada perangkat dengan spesifikasi sebagai berikut:

\begin{table}[H]
\centering
\caption{Spesifikasi Perangkat Keras}
\label{tab:spesifikasi-hardware}
\begin{tabular}{|l|l|}
\hline
\textbf{Komponen} & \textbf{Spesifikasi} \\
\hline
Processor & Intel Core i5 / AMD Ryzen 5 atau lebih tinggi \\
\hline
RAM & Minimal 8 GB \\
\hline
Storage & SSD 256 GB atau lebih \\
\hline
Display & Resolusi minimal 1920 x 1080 \\
\hline
\end{tabular}
\end{table}

\subsection{Spesifikasi Perangkat Lunak}
\begin{table}[H]
\centering
\caption{Spesifikasi Perangkat Lunak}
\label{tab:spesifikasi-software}
\begin{tabular}{|l|l|l|}
\hline
\textbf{Software} & \textbf{Versi} & \textbf{Keterangan} \\
\hline
Operating System & Linux/Windows/macOS & Development \& Production \\
\hline
PHP & 8.4.x & Backend Runtime \\
\hline
Laravel Framework & 12.x & PHP Framework \\
\hline
Node.js & 20.x LTS & Frontend Build Tool \\
\hline
SQLite/MySQL & 3.x / 8.x & Database Management \\
\hline
Composer & 2.x & PHP Dependency Manager \\
\hline
NPM & 10.x & Node Package Manager \\
\hline
Visual Studio Code & Latest & Code Editor \\
\hline
Git & Latest & Version Control \\
\hline
\end{tabular}
\end{table}

\subsection{Teknologi yang Digunakan}
\begin{table}[H]
\centering
\caption{Stack Teknologi SISRI-BPI}
\label{tab:tech-stack}
\begin{tabular}{|l|l|p{7cm}|}
\hline
\textbf{Layer} & \textbf{Teknologi} & \textbf{Fungsi} \\
\hline
Frontend & TailwindCSS & Utility-first CSS framework untuk styling \\
\hline
Frontend & Alpine.js & Lightweight JavaScript framework untuk interaktivitas \\
\hline
Frontend & SweetAlert2 & Library untuk dialog dan notifikasi \\
\hline
Backend & Laravel 12 & PHP framework dengan arsitektur MVC \\
\hline
Backend & Laravel Breeze & Authentication scaffolding \\
\hline
Backend & Spatie Permission & Role \& permission management \\
\hline
Database & SQLite/MySQL & Relational database management system \\
\hline
Build Tool & Vite & Frontend asset bundler \\
\hline
Server & Laravel Octane & High-performance application server \\
\hline
\end{tabular}
\end{table}

% ============================================
\section{Struktur Direktori Proyek}

Struktur direktori proyek SISRI-BPI mengikuti konvensi Laravel dengan beberapa penyesuaian:

\begin{lstlisting}[language=bash, caption={Struktur Direktori SISRI-BPI}, label={lst:struktur-direktori}]
SISRI-BPI/
├── app/
│   ├── Http/
│   │   ├── Controllers/
│   │   │   ├── Admin/
│   │   │   ├── Dosen/
│   │   │   ├── Koordinator/
│   │   │   └── Mahasiswa/
│   │   ├── Middleware/
│   │   └── Requests/
│   ├── Models/
│   │   ├── User.php
│   │   ├── Mahasiswa.php
│   │   ├── Dosen.php
│   │   ├── TopikSkripsi.php
│   │   ├── Bimbingan.php
│   │   ├── PendaftaranSidang.php
│   │   ├── PelaksanaanSidang.php
│   │   └── ...
│   └── Providers/
├── database/
│   ├── migrations/
│   └── seeders/
├── resources/
│   ├── views/
│   │   ├── admin/
│   │   ├── dosen/
│   │   ├── koordinator/
│   │   ├── mahasiswa/
│   │   ├── layouts/
│   │   └── components/
│   ├── css/
│   └── js/
├── routes/
│   ├── web.php
│   └── auth.php
└── public/
\end{lstlisting}

% ============================================
\section{Implementasi Database}

\subsection{Implementasi Migration}
Laravel menggunakan migration untuk membuat dan mengelola struktur database. Berikut contoh implementasi migration untuk tabel \texttt{topik\_skripsi}:

\begin{lstlisting}[language=PHP, caption={Migration Tabel Topik Skripsi}, label={lst:migration-topik}]
<?php

use Illuminate\Database\Migrations\Migration;
use Illuminate\Database\Schema\Blueprint;
use Illuminate\Support\Facades\Schema;

return new class extends Migration
{
    public function up(): void
    {
        Schema::create('topik_skripsi', function (Blueprint $table) {
            $table->id();
            $table->foreignId('mahasiswa_id')
                  ->constrained('mahasiswa')
                  ->cascadeOnDelete();
            $table->foreignId('bidang_minat_id')
                  ->constrained('bidang_minat')
                  ->cascadeOnDelete();
            $table->string('judul', 500);
            $table->string('file_proposal', 255)->nullable();
            $table->enum('status', ['menunggu', 'diterima', 'ditolak'])
                  ->default('menunggu');
            $table->text('catatan')->nullable();
            $table->timestamps();
        });
    }

    public function down(): void
    {
        Schema::dropIfExists('topik_skripsi');
    }
};
\end{lstlisting}

\subsection{Implementasi Model}
Model Eloquent digunakan untuk berinteraksi dengan database. Berikut implementasi model \texttt{TopikSkripsi}:

\begin{lstlisting}[language=PHP, caption={Model TopikSkripsi}, label={lst:model-topik}]
<?php

namespace App\Models;

use Illuminate\Database\Eloquent\Model;
use Illuminate\Database\Eloquent\Relations\BelongsTo;
use Illuminate\Database\Eloquent\Relations\HasMany;

class TopikSkripsi extends Model
{
    protected $table = 'topik_skripsi';

    protected $fillable = [
        'mahasiswa_id',
        'bidang_minat_id',
        'judul',
        'file_proposal',
        'status',
        'catatan',
    ];

    public function mahasiswa(): BelongsTo
    {
        return $this->belongsTo(Mahasiswa::class);
    }

    public function bidangMinat(): BelongsTo
    {
        return $this->belongsTo(BidangMinat::class);
    }

    public function usulanPembimbing(): HasMany
    {
        return $this->hasMany(UsulanPembimbing::class, 'topik_id');
    }

    public function bimbingan(): HasMany
    {
        return $this->hasMany(Bimbingan::class, 'topik_id');
    }

    public function pendaftaranSidang(): HasMany
    {
        return $this->hasMany(PendaftaranSidang::class, 'topik_id');
    }

    // Accessor untuk mendapatkan pembimbing yang sudah diterima
    public function getPembimbing1Attribute()
    {
        return $this->usulanPembimbing()
            ->where('urutan', 1)
            ->where('status', 'diterima')
            ->first()?->dosen;
    }

    public function getPembimbing2Attribute()
    {
        return $this->usulanPembimbing()
            ->where('urutan', 2)
            ->where('status', 'diterima')
            ->first()?->dosen;
    }
}
\end{lstlisting}

% ============================================
\section{Implementasi Controller}

\subsection{Controller Bimbingan Mahasiswa}
Berikut implementasi controller untuk mengelola bimbingan dari sisi mahasiswa:

\begin{lstlisting}[language=PHP, caption={BimbinganController untuk Mahasiswa}, label={lst:controller-bimbingan}]
<?php

namespace App\Http\Controllers\Mahasiswa;

use App\Http\Controllers\Controller;
use App\Models\Bimbingan;
use App\Models\BimbinganHistory;
use Illuminate\Http\Request;
use Illuminate\Support\Facades\Auth;
use Illuminate\Support\Facades\Storage;

class BimbinganController extends Controller
{
    public function index()
    {
        $mahasiswa = Auth::user()->mahasiswa;
        $topik = $mahasiswa->topikSkripsi()
            ->where('status', 'diterima')
            ->first();

        if (!$topik) {
            return redirect()->route('mahasiswa.topik.index')
                ->with('error', 'Anda belum memiliki topik yang disetujui');
        }

        $bimbingan = Bimbingan::where('topik_id', $topik->id)
            ->with(['dosen', 'histories'])
            ->orderBy('created_at', 'desc')
            ->paginate(10);

        return view('mahasiswa.bimbingan.index', compact('bimbingan', 'topik'));
    }

    public function store(Request $request)
    {
        $request->validate([
            'dosen_id' => 'required|exists:dosen,id',
            'jenis' => 'required|in:proposal,skripsi',
            'pokok_bimbingan' => 'required|string|max:1000',
            'pesan_mahasiswa' => 'nullable|string|max:2000',
            'file_bimbingan' => 'nullable|file|mimes:pdf,doc,docx|max:10240',
        ]);

        $mahasiswa = Auth::user()->mahasiswa;
        $topik = $mahasiswa->topikSkripsi()
            ->where('status', 'diterima')
            ->firstOrFail();

        $filePath = null;
        if ($request->hasFile('file_bimbingan')) {
            $filePath = $request->file('file_bimbingan')
                ->store('bimbingan', 'public');
        }

        $bimbingan = Bimbingan::create([
            'topik_id' => $topik->id,
            'dosen_id' => $request->dosen_id,
            'jenis' => $request->jenis,
            'pokok_bimbingan' => $request->pokok_bimbingan,
            'pesan_mahasiswa' => $request->pesan_mahasiswa,
            'file_bimbingan' => $filePath,
            'status' => 'menunggu',
            'tanggal_bimbingan' => now(),
        ]);

        // Catat history
        BimbinganHistory::create([
            'bimbingan_id' => $bimbingan->id,
            'status' => 'menunggu',
            'aksi' => 'submit',
            'catatan' => 'Mahasiswa mengajukan bimbingan',
            'oleh' => $mahasiswa->nama,
        ]);

        return redirect()->route('mahasiswa.bimbingan.index')
            ->with('success', 'Bimbingan berhasil diajukan');
    }
}
\end{lstlisting}

\subsection{Controller Penjadwalan Otomatis}
Implementasi fitur penjadwalan sidang otomatis oleh koordinator:

\begin{lstlisting}[language=PHP, caption={JadwalOtomatisController}, label={lst:controller-jadwal}]
<?php

namespace App\Http\Controllers\Koordinator;

use App\Http\Controllers\Controller;
use App\Models\PendaftaranSidang;
use App\Models\PelaksanaanSidang;
use App\Models\PengujiSidang;
use App\Models\Ruangan;
use Illuminate\Http\Request;
use Carbon\Carbon;

class JadwalOtomatisController extends Controller
{
    public function generate(Request $request)
    {
        $request->validate([
            'jadwal_sidang_id' => 'required|exists:jadwal_sidang,id',
            'tanggal_mulai' => 'required|date',
            'tanggal_selesai' => 'required|date|after_or_equal:tanggal_mulai',
        ]);

        // Ambil pendaftaran yang menunggu
        $pendaftarans = PendaftaranSidang::where('jadwal_sidang_id', $request->jadwal_sidang_id)
            ->where('status_koordinator', 'menunggu')
            ->where('status_pembimbing_1', 'disetujui')
            ->where('status_pembimbing_2', 'disetujui')
            ->with(['topik.usulanPembimbing.dosen'])
            ->get();

        // Ambil ruangan aktif
        $ruangans = Ruangan::where('is_active', true)->get();
        
        // Slot waktu (08:00 - 16:00, @1.5 jam)
        $slots = ['08:00', '09:30', '11:00', '13:00', '14:30'];
        
        $scheduled = 0;
        $currentDate = Carbon::parse($request->tanggal_mulai);
        $endDate = Carbon::parse($request->tanggal_selesai);

        foreach ($pendaftarans as $pendaftaran) {
            // Cari slot kosong
            $slotFound = false;
            
            while ($currentDate <= $endDate && !$slotFound) {
                // Skip weekend
                if ($currentDate->isWeekend()) {
                    $currentDate->addDay();
                    continue;
                }

                foreach ($ruangans as $ruangan) {
                    foreach ($slots as $slot) {
                        $waktu = $currentDate->copy()
                            ->setTimeFromTimeString($slot);

                        // Cek konflik ruangan
                        $conflict = PelaksanaanSidang::where('tanggal_sidang', $waktu)
                            ->where('tempat', 'LIKE', '%' . $ruangan->nama . '%')
                            ->exists();

                        if (!$conflict) {
                            // Buat pelaksanaan sidang
                            $pelaksanaan = PelaksanaanSidang::create([
                                'pendaftaran_sidang_id' => $pendaftaran->id,
                                'tanggal_sidang' => $waktu,
                                'tempat' => $ruangan->nama . ' - ' . $ruangan->lokasi,
                                'status' => 'dijadwalkan',
                            ]);

                            // Set penguji dari pembimbing
                            $this->assignPenguji($pelaksanaan, $pendaftaran);

                            // Update status pendaftaran
                            $pendaftaran->update([
                                'status_koordinator' => 'disetujui'
                            ]);

                            $scheduled++;
                            $slotFound = true;
                            break 2;
                        }
                    }
                }
                $currentDate->addDay();
            }
        }

        return redirect()->back()
            ->with('success', "Berhasil menjadwalkan {$scheduled} sidang");
    }

    private function assignPenguji($pelaksanaan, $pendaftaran)
    {
        $pembimbings = $pendaftaran->topik->usulanPembimbing()
            ->where('status', 'diterima')
            ->orderBy('urutan')
            ->get();

        foreach ($pembimbings as $p) {
            PengujiSidang::create([
                'pelaksanaan_sidang_id' => $pelaksanaan->id,
                'dosen_id' => $p->dosen_id,
                'role' => 'pembimbing_' . $p->urutan,
            ]);
        }
    }
}
\end{lstlisting}

% ============================================
\section{Implementasi View}

\subsection{Layout Utama}
Sistem menggunakan Blade templating engine dengan layout berbasis komponen:

\begin{lstlisting}[language=HTML, caption={Layout Utama Aplikasi}, label={lst:layout-app}]
<!DOCTYPE html>
<html lang="{{ str_replace('_', '-', app()->getLocale()) }}">
<head>
    <meta charset="utf-8">
    <meta name="viewport" content="width=device-width, initial-scale=1">
    <meta name="csrf-token" content="{{ csrf_token() }}">

    <title>{{ config('app.name', 'SISRI-BPI') }}</title>

    <!-- Fonts -->
    <link rel="preconnect" href="https://fonts.bunny.net">
    <link href="https://fonts.bunny.net/css?family=figtree:400,500,600&display=swap" 
          rel="stylesheet" />

    <!-- Scripts -->
    @vite(['resources/css/app.css', 'resources/js/app.js'])
</head>
<body class="font-sans antialiased">
    <div class="min-h-screen bg-gray-100">
        <!-- Sidebar -->
        @include('layouts.sidebar')

        <!-- Page Content -->
        <div class="lg:pl-64">
            <!-- Top Navigation -->
            @include('layouts.navigation')

            <!-- Main Content -->
            <main class="py-6 px-4 sm:px-6 lg:px-8">
                <!-- Flash Messages -->
                @if(session('success'))
                    <x-alert type="success" :message="session('success')" />
                @endif

                @if(session('error'))
                    <x-alert type="error" :message="session('error')" />
                @endif

                {{ $slot }}
            </main>
        </div>
    </div>

    <!-- SweetAlert2 -->
    <script src="https://cdn.jsdelivr.net/npm/sweetalert2@11"></script>
    @stack('scripts')
</body>
</html>
\end{lstlisting}

\subsection{Halaman Daftar Bimbingan}
\begin{lstlisting}[language=HTML, caption={View Daftar Bimbingan Mahasiswa}, label={lst:view-bimbingan}]
<x-app-layout>
    <x-slot name="header">
        <h2 class="text-xl font-semibold text-gray-800">
            Bimbingan Skripsi
        </h2>
    </x-slot>

    <div class="bg-white rounded-lg shadow-md p-6">
        <!-- Header dengan tombol tambah -->
        <div class="flex justify-between items-center mb-6">
            <h3 class="text-lg font-medium">Riwayat Bimbingan</h3>
            <a href="{{ route('mahasiswa.bimbingan.create') }}"
               class="bg-blue-600 hover:bg-blue-700 text-white 
                      px-4 py-2 rounded-lg transition-colors">
                + Ajukan Bimbingan
            </a>
        </div>

        <!-- Tabel Bimbingan -->
        <div class="overflow-x-auto">
            <table class="min-w-full divide-y divide-gray-200">
                <thead class="bg-gray-50">
                    <tr>
                        <th class="px-6 py-3 text-left text-xs font-medium 
                                   text-gray-500 uppercase">No</th>
                        <th class="px-6 py-3 text-left text-xs font-medium 
                                   text-gray-500 uppercase">Tanggal</th>
                        <th class="px-6 py-3 text-left text-xs font-medium 
                                   text-gray-500 uppercase">Pembimbing</th>
                        <th class="px-6 py-3 text-left text-xs font-medium 
                                   text-gray-500 uppercase">Pokok Bahasan</th>
                        <th class="px-6 py-3 text-left text-xs font-medium 
                                   text-gray-500 uppercase">Status</th>
                        <th class="px-6 py-3 text-left text-xs font-medium 
                                   text-gray-500 uppercase">Aksi</th>
                    </tr>
                </thead>
                <tbody class="divide-y divide-gray-200">
                    @forelse($bimbingan as $index => $item)
                    <tr class="hover:bg-gray-50">
                        <td class="px-6 py-4 text-sm">
                            {{ $bimbingan->firstItem() + $index }}
                        </td>
                        <td class="px-6 py-4 text-sm">
                            {{ $item->tanggal_bimbingan->format('d M Y') }}
                        </td>
                        <td class="px-6 py-4 text-sm">
                            {{ $item->dosen->nama }}
                        </td>
                        <td class="px-6 py-4 text-sm">
                            {{ Str::limit($item->pokok_bimbingan, 50) }}
                        </td>
                        <td class="px-6 py-4">
                            <x-status-badge :status="$item->status" />
                        </td>
                        <td class="px-6 py-4 text-sm">
                            <a href="{{ route('mahasiswa.bimbingan.show', $item) }}"
                               class="text-blue-600 hover:text-blue-800">
                                Detail
                            </a>
                        </td>
                    </tr>
                    @empty
                    <tr>
                        <td colspan="6" class="px-6 py-4 text-center text-gray-500">
                            Belum ada data bimbingan
                        </td>
                    </tr>
                    @endforelse
                </tbody>
            </table>
        </div>

        <!-- Pagination -->
        <div class="mt-4">
            {{ $bimbingan->links() }}
        </div>
    </div>
</x-app-layout>
\end{lstlisting}

% ============================================
\section{Implementasi Autentikasi dan Otorisasi}

\subsection{Middleware Role-Based Access Control}
\begin{lstlisting}[language=PHP, caption={Middleware RoleMiddleware}, label={lst:middleware-role}]
<?php

namespace App\Http\Middleware;

use Closure;
use Illuminate\Http\Request;

class RoleMiddleware
{
    public function handle(Request $request, Closure $next, ...$roles)
    {
        if (!$request->user()) {
            return redirect()->route('login');
        }

        if (!$request->user()->hasAnyRole($roles)) {
            abort(403, 'Unauthorized action.');
        }

        return $next($request);
    }
}
\end{lstlisting}

\subsection{Konfigurasi Route dengan Middleware}
\begin{lstlisting}[language=PHP, caption={Konfigurasi Route dengan Role Middleware}, label={lst:route-middleware}]
<?php

use Illuminate\Support\Facades\Route;

// Route Mahasiswa
Route::middleware(['auth', 'role:mahasiswa'])
    ->prefix('mahasiswa')
    ->name('mahasiswa.')
    ->group(function () {
        Route::get('/dashboard', [DashboardController::class, 'index'])
            ->name('dashboard');
        Route::resource('topik', TopikController::class);
        Route::resource('bimbingan', BimbinganController::class);
        Route::resource('sidang', SidangController::class);
    });

// Route Dosen
Route::middleware(['auth', 'role:dosen'])
    ->prefix('dosen')
    ->name('dosen.')
    ->group(function () {
        Route::get('/dashboard', [DosenDashboardController::class, 'index'])
            ->name('dashboard');
        Route::resource('bimbingan', DosenBimbinganController::class);
        Route::resource('nilai', NilaiController::class);
    });

// Route Koordinator
Route::middleware(['auth', 'role:koordinator'])
    ->prefix('koordinator')
    ->name('koordinator.')
    ->group(function () {
        Route::get('/dashboard', [KoordinatorDashboardController::class, 'index'])
            ->name('dashboard');
        Route::resource('topik', KoordinatorTopikController::class);
        Route::post('/jadwal-otomatis', [JadwalOtomatisController::class, 'generate'])
            ->name('jadwal.generate');
    });

// Route Admin
Route::middleware(['auth', 'role:admin'])
    ->prefix('admin')
    ->name('admin.')
    ->group(function () {
        Route::get('/dashboard', [AdminDashboardController::class, 'index'])
            ->name('dashboard');
        Route::resource('users', UserController::class);
        Route::resource('dosen', DosenController::class);
        Route::resource('mahasiswa', MahasiswaController::class);
    });
\end{lstlisting}

% ============================================
\section{Implementasi Fitur Utama}

\subsection{Fitur Upload File}
\begin{lstlisting}[language=PHP, caption={Implementasi Upload File}, label={lst:upload-file}]
<?php

// Di Controller
public function uploadProposal(Request $request)
{
    $request->validate([
        'file_proposal' => 'required|file|mimes:pdf|max:5120', // Max 5MB
    ]);

    if ($request->hasFile('file_proposal')) {
        // Hapus file lama jika ada
        if ($this->topik->file_proposal) {
            Storage::disk('public')->delete($this->topik->file_proposal);
        }

        // Upload file baru
        $file = $request->file('file_proposal');
        $filename = 'proposal_' . $this->topik->mahasiswa->nim . '_' 
                  . time() . '.' . $file->getClientOriginalExtension();
        
        $path = $file->storeAs('proposals', $filename, 'public');

        $this->topik->update(['file_proposal' => $path]);
    }

    return back()->with('success', 'File proposal berhasil diupload');
}
\end{lstlisting}

\subsection{Fitur Notifikasi Real-time}
\begin{lstlisting}[language=JavaScript, caption={Implementasi Notifikasi dengan SweetAlert2}, label={lst:notifikasi}]
// resources/js/notifications.js

// Flash message handler
document.addEventListener('DOMContentLoaded', function() {
    // Success notification
    const successMsg = document.querySelector('[data-success-message]');
    if (successMsg) {
        Swal.fire({
            icon: 'success',
            title: 'Berhasil!',
            text: successMsg.dataset.successMessage,
            timer: 3000,
            showConfirmButton: false
        });
    }

    // Error notification
    const errorMsg = document.querySelector('[data-error-message]');
    if (errorMsg) {
        Swal.fire({
            icon: 'error',
            title: 'Gagal!',
            text: errorMsg.dataset.errorMessage
        });
    }
});

// Confirm delete
function confirmDelete(formId) {
    Swal.fire({
        title: 'Apakah Anda yakin?',
        text: "Data yang dihapus tidak dapat dikembalikan!",
        icon: 'warning',
        showCancelButton: true,
        confirmButtonColor: '#d33',
        cancelButtonColor: '#3085d6',
        confirmButtonText: 'Ya, hapus!',
        cancelButtonText: 'Batal'
    }).then((result) => {
        if (result.isConfirmed) {
            document.getElementById(formId).submit();
        }
    });
}
\end{lstlisting}

% ============================================
\section{Hasil Implementasi}

\subsection{Halaman Login}
Halaman login merupakan halaman pertama yang diakses pengguna untuk masuk ke sistem. Tampilan halaman login dapat dilihat pada Gambar \ref{fig:impl-login}.

\begin{figure}[H]
\centering
\fbox{\includegraphics[width=0.8\textwidth]{images/login.png}}
\caption{Implementasi Halaman Login}
\label{fig:impl-login}
\end{figure}

\subsection{Dashboard Mahasiswa}
Dashboard mahasiswa menampilkan ringkasan informasi terkait proses skripsi mahasiswa yang bersangkutan.

\begin{figure}[H]
\centering
\fbox{\includegraphics[width=0.9\textwidth]{images/dashboard-mahasiswa.png}}
\caption{Implementasi Dashboard Mahasiswa}
\label{fig:impl-dashboard-mhs}
\end{figure}

\subsection{Halaman Bimbingan}
Halaman bimbingan memungkinkan mahasiswa untuk mengajukan dan memantau proses bimbingan dengan dosen pembimbing.

\begin{figure}[H]
\centering
\fbox{\includegraphics[width=0.9\textwidth]{images/bimbingan.png}}
\caption{Implementasi Halaman Bimbingan}
\label{fig:impl-bimbingan}
\end{figure}

\subsection{Halaman Penjadwalan Sidang}
Koordinator dapat melakukan penjadwalan sidang secara otomatis melalui halaman ini.

\begin{figure}[H]
\centering
\fbox{\includegraphics[width=0.9\textwidth]{images/jadwal-sidang.png}}
\caption{Implementasi Halaman Penjadwalan Sidang}
\label{fig:impl-jadwal}
\end{figure}

\subsection{Halaman Input Nilai}
Dosen penguji dapat memasukkan nilai sidang melalui form yang telah disediakan.

\begin{figure}[H]
\centering
\fbox{\includegraphics[width=0.9\textwidth]{images/input-nilai.png}}
\caption{Implementasi Halaman Input Nilai}
\label{fig:impl-nilai}
\end{figure}

% ============================================
\section{Pengujian Sistem}

\subsection{Pengujian Black-Box}
Pengujian black-box dilakukan untuk memastikan fungsionalitas sistem berjalan sesuai dengan kebutuhan.

\begin{table}[H]
\centering
\caption{Hasil Pengujian Black-Box}
\label{tab:pengujian-blackbox}
\begin{tabular}{|c|p{4cm}|p{4cm}|p{3cm}|c|}
\hline
\textbf{No} & \textbf{Skenario Pengujian} & \textbf{Hasil yang Diharapkan} & \textbf{Hasil Pengujian} & \textbf{Status} \\
\hline
1 & Login dengan kredensial valid & Masuk ke dashboard sesuai role & Sesuai harapan & \checkmark \\
\hline
2 & Login dengan kredensial invalid & Menampilkan pesan error & Sesuai harapan & \checkmark \\
\hline
3 & Mahasiswa mengajukan topik & Topik tersimpan dengan status menunggu & Sesuai harapan & \checkmark \\
\hline
4 & Koordinator menyetujui topik & Status topik berubah menjadi diterima & Sesuai harapan & \checkmark \\
\hline
5 & Mahasiswa mengajukan bimbingan & Bimbingan tersimpan dan notifikasi terkirim & Sesuai harapan & \checkmark \\
\hline
6 & Dosen merespon bimbingan & Status bimbingan terupdate & Sesuai harapan & \checkmark \\
\hline
7 & Koordinator generate jadwal otomatis & Jadwal sidang terbuat tanpa konflik & Sesuai harapan & \checkmark \\
\hline
8 & Dosen input nilai sidang & Nilai tersimpan di database & Sesuai harapan & \checkmark \\
\hline
\end{tabular}
\end{table}

\subsection{Pengujian Responsivitas}
Sistem diuji pada berbagai ukuran layar untuk memastikan tampilan responsif.

\begin{table}[H]
\centering
\caption{Hasil Pengujian Responsivitas}
\label{tab:pengujian-responsif}
\begin{tabular}{|l|c|c|c|}
\hline
\textbf{Device} & \textbf{Resolusi} & \textbf{Browser} & \textbf{Status} \\
\hline
Desktop & 1920x1080 & Chrome, Firefox & \checkmark \\
\hline
Laptop & 1366x768 & Chrome, Edge & \checkmark \\
\hline
Tablet & 768x1024 & Safari, Chrome & \checkmark \\
\hline
Mobile & 375x667 & Chrome, Safari & \checkmark \\
\hline
\end{tabular}
\end{table}

\subsection{Pengujian Performa}
Pengujian performa dilakukan untuk mengukur waktu respon sistem.

\begin{table}[H]
\centering
\caption{Hasil Pengujian Performa}
\label{tab:pengujian-performa}
\begin{tabular}{|l|c|c|}
\hline
\textbf{Halaman} & \textbf{Waktu Muat (ms)} & \textbf{Status} \\
\hline
Login & 245 & Baik \\
\hline
Dashboard & 312 & Baik \\
\hline
List Bimbingan & 287 & Baik \\
\hline
Form Pengajuan & 198 & Baik \\
\hline
Generate Jadwal Otomatis & 1245 & Cukup \\
\hline
\end{tabular}
\end{table}
