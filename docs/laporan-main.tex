% ============================================
% SISRI-BPI Laporan Skripsi - Main Document
% Universitas Trunojoyo Madura
% ============================================

\documentclass[12pt,a4paper]{report}

% ============================================
% PACKAGES
% ============================================
\usepackage[utf8]{inputenc}
\usepackage[indonesian]{babel}
\usepackage{graphicx}
\usepackage{float}
\usepackage{geometry}
\usepackage{setspace}
\usepackage{titlesec}
\usepackage{tocloft}
\usepackage{hyperref}
\usepackage{listings}
\usepackage{xcolor}
\usepackage{tikz}
\usepackage{pgf-umlsd} % untuk sequence diagram
\usepackage{array}
\usepackage{tabularx}
\usepackage{booktabs}
\usepackage{multirow}
\usepackage{caption}
\usepackage{subcaption}
\usepackage{amsmath}
\usepackage{amssymb}
\usepackage{fancyhdr}
\usepackage{indentfirst}
\usepackage{enumitem}

% ============================================
% TikZ Libraries
% ============================================
\usetikzlibrary{shapes.geometric, arrows, positioning, fit, backgrounds, calc}

% ============================================
% PAGE SETUP
% ============================================
\geometry{
    a4paper,
    left=4cm,
    right=3cm,
    top=3cm,
    bottom=3cm
}

% Line spacing 1.5
\onehalfspacing

% ============================================
% HEADER & FOOTER
% ============================================
\pagestyle{fancy}
\fancyhf{}
\fancyhead[R]{\thepage}
\renewcommand{\headrulewidth}{0pt}
\renewcommand{\footrulewidth}{0pt}

% ============================================
% CHAPTER & SECTION FORMATTING
% ============================================
\titleformat{\chapter}[display]
    {\normalfont\Large\bfseries\centering}
    {BAB \thechapter}
    {0pt}
    {\Large\uppercase}
\titlespacing*{\chapter}{0pt}{-30pt}{20pt}

\titleformat{\section}
    {\normalfont\large\bfseries}
    {\thesection}
    {1em}
    {}

\titleformat{\subsection}
    {\normalfont\normalsize\bfseries}
    {\thesubsection}
    {1em}
    {}

% ============================================
% CODE LISTING STYLE
% ============================================
\definecolor{codegreen}{rgb}{0,0.6,0}
\definecolor{codegray}{rgb}{0.5,0.5,0.5}
\definecolor{codepurple}{rgb}{0.58,0,0.82}
\definecolor{backcolour}{rgb}{0.97,0.97,0.97}

\lstdefinestyle{mystyle}{
    backgroundcolor=\color{backcolour},
    commentstyle=\color{codegreen},
    keywordstyle=\color{blue},
    numberstyle=\tiny\color{codegray},
    stringstyle=\color{codepurple},
    basicstyle=\ttfamily\footnotesize,
    breakatwhitespace=false,
    breaklines=true,
    captionpos=b,
    keepspaces=true,
    numbers=left,
    numbersep=5pt,
    showspaces=false,
    showstringspaces=false,
    showtabs=false,
    tabsize=2,
    frame=single,
    rulecolor=\color{gray!30}
}
\lstset{style=mystyle}

% PHP Language Definition
\lstdefinelanguage{PHP}{
    keywords={class, function, return, if, else, elseif, foreach, for, while, do, switch, case, break, continue, public, private, protected, static, final, abstract, extends, implements, namespace, use, new, true, false, null, try, catch, throw, interface, trait},
    keywordstyle=\color{blue}\bfseries,
    ndkeywords={Route, Schema, DB, Auth, Request, Response, View, Model, Controller},
    ndkeywordstyle=\color{purple}\bfseries,
    sensitive=true,
    comment=[l]{//},
    morecomment=[s]{/*}{*/},
    commentstyle=\color{codegreen}\ttfamily,
    stringstyle=\color{red}\ttfamily,
    morestring=[b]',
    morestring=[b]"
}

% ============================================
% HYPERREF SETUP
% ============================================
\hypersetup{
    colorlinks=true,
    linkcolor=black,
    filecolor=magenta,
    urlcolor=blue,
    citecolor=black,
}

% ============================================
% TABLE OF CONTENTS CUSTOMIZATION
% ============================================
\renewcommand{\contentsname}{DAFTAR ISI}
\renewcommand{\listfigurename}{DAFTAR GAMBAR}
\renewcommand{\listtablename}{DAFTAR TABEL}

% ============================================
% CUSTOM COMMANDS
% ============================================
\newcommand{\judul}{Sistem Informasi Skripsi Berbasis Web (SISRI-BPI)}
\newcommand{\penulis}{[Nama Mahasiswa]}
\newcommand{\nim}{[NIM]}
\newcommand{\prodi}{Teknik Informatika}
\newcommand{\fakultas}{Fakultas Teknik}
\newcommand{\universitas}{Universitas Trunojoyo Madura}
\newcommand{\tahun}{2025}

% ============================================
% DOCUMENT BEGIN
% ============================================
\begin{document}

% ============================================
% FRONT MATTER
% ============================================
\pagenumbering{roman}

% Halaman Judul
\begin{titlepage}
    \centering
    \vspace*{1cm}
    
    {\Large\bfseries SKRIPSI}\\[0.5cm]
    
    \vspace{1cm}
    
    {\LARGE\bfseries\judul}\\[1.5cm]
    
    \vspace{1cm}
    
    \includegraphics[width=4cm]{images/logo-utm.png}\\[1cm]
    
    \vspace{1cm}
    
    {\large Oleh:}\\[0.3cm]
    {\large\bfseries\penulis}\\
    {\large NIM. \nim}\\[1.5cm]
    
    \vspace{1cm}
    
    {\large\bfseries PROGRAM STUDI \MakeUppercase{\prodi}}\\
    {\large\bfseries\MakeUppercase{\fakultas}}\\
    {\large\bfseries\MakeUppercase{\universitas}}\\
    {\large\bfseries\tahun}
    
\end{titlepage}

% Daftar Isi
\tableofcontents
\newpage

% Daftar Gambar
\listoffigures
\newpage

% Daftar Tabel
\listoftables
\newpage

% ============================================
% MAIN MATTER
% ============================================
\pagenumbering{arabic}

% Include BAB 4 - Analisis dan Perancangan
% ============================================
% BAB 4 - ANALISIS DAN PERANCANGAN SISTEM
% SISRI-BPI (Sistem Informasi Skripsi)
% Universitas Trunojoyo Madura
% ============================================

\chapter{ANALISIS DAN PERANCANGAN SISTEM}

\section{Analisis Sistem}

\subsection{Analisis Sistem yang Berjalan}
Sistem pengelolaan skripsi yang saat ini berjalan di Program Studi Teknik Informatika Fakultas Teknik Universitas Trunojoyo Madura masih menggunakan metode konvensional dengan beberapa permasalahan sebagai berikut:

\begin{enumerate}
    \item Pengajuan topik skripsi dilakukan secara manual melalui formulir fisik
    \item Proses persetujuan pembimbing membutuhkan waktu yang lama karena harus bertemu langsung
    \item Penjadwalan sidang dilakukan secara manual oleh koordinator
    \item Dokumentasi bimbingan tidak terstruktur dengan baik
    \item Tidak ada sistem notifikasi otomatis untuk mahasiswa dan dosen
\end{enumerate}

\subsection{Analisis Kebutuhan Sistem}

\subsubsection{Kebutuhan Fungsional}
Berdasarkan analisis permasalahan, sistem yang akan dibangun harus memenuhi kebutuhan fungsional sebagai berikut:

\begin{table}[H]
\centering
\caption{Kebutuhan Fungsional Sistem}
\label{tab:kebutuhan-fungsional}
\begin{tabular}{|c|l|p{8cm}|}
\hline
\textbf{No} & \textbf{Kode} & \textbf{Deskripsi} \\
\hline
1 & KF-01 & Sistem dapat mengelola data pengguna (admin, mahasiswa, dosen, koordinator) \\
\hline
2 & KF-02 & Sistem dapat mengelola pengajuan topik skripsi oleh mahasiswa \\
\hline
3 & KF-03 & Sistem dapat mengelola usulan dan persetujuan pembimbing \\
\hline
4 & KF-04 & Sistem dapat mencatat dan mengelola proses bimbingan \\
\hline
5 & KF-05 & Sistem dapat mengelola pendaftaran seminar proposal dan sidang skripsi \\
\hline
6 & KF-06 & Sistem dapat melakukan penjadwalan sidang secara otomatis \\
\hline
7 & KF-07 & Sistem dapat mengelola penilaian sidang \\
\hline
8 & KF-08 & Sistem dapat mengelola revisi pasca sidang \\
\hline
9 & KF-09 & Sistem dapat menghasilkan laporan dan statistik \\
\hline
\end{tabular}
\end{table}

\subsubsection{Kebutuhan Non-Fungsional}
\begin{table}[H]
\centering
\caption{Kebutuhan Non-Fungsional Sistem}
\label{tab:kebutuhan-non-fungsional}
\begin{tabular}{|c|l|p{8cm}|}
\hline
\textbf{No} & \textbf{Aspek} & \textbf{Deskripsi} \\
\hline
1 & Keamanan & Sistem menggunakan autentikasi berbasis session dan role-based access control \\
\hline
2 & Performa & Sistem dapat menangani minimal 100 pengguna secara bersamaan \\
\hline
3 & Usability & Antarmuka sistem mudah digunakan dan responsif \\
\hline
4 & Reliability & Sistem memiliki uptime minimal 99\% \\
\hline
5 & Scalability & Sistem dapat dikembangkan untuk menambah fitur baru \\
\hline
\end{tabular}
\end{table}

\subsection{Analisis Pengguna}
Sistem SISRI-BPI memiliki empat jenis pengguna dengan hak akses yang berbeda:

\begin{table}[H]
\centering
\caption{Analisis Pengguna Sistem}
\label{tab:analisis-pengguna}
\begin{tabular}{|c|l|p{9cm}|}
\hline
\textbf{No} & \textbf{Pengguna} & \textbf{Hak Akses} \\
\hline
1 & Admin & Mengelola seluruh data master, pengguna, dan konfigurasi sistem \\
\hline
2 & Mahasiswa & Mengajukan topik, melakukan bimbingan, mendaftar sidang, melihat nilai \\
\hline
3 & Dosen & Menyetujui pembimbingan, memberikan feedback bimbingan, menilai sidang \\
\hline
4 & Koordinator & Menyetujui topik, menjadwalkan sidang, mengelola periode akademik \\
\hline
\end{tabular}
\end{table}

% ============================================
\section{Perancangan Sistem}

\subsection{Arsitektur Sistem}
Sistem SISRI-BPI dibangun menggunakan arsitektur \textit{Model-View-Controller} (MVC) dengan framework Laravel. Arsitektur sistem ditunjukkan pada Gambar \ref{fig:arsitektur-sistem}.

\begin{figure}[H]
\centering
\begin{tikzpicture}[
    node distance=1.5cm,
    box/.style={rectangle, draw, rounded corners, minimum width=3cm, minimum height=1cm, align=center, fill=blue!10},
    arrow/.style={->, thick}
]
    % Client Layer
    \node[box, fill=green!20] (browser) {Web Browser\\(Client)};
    
    % Presentation Layer
    \node[box, below=of browser, fill=yellow!20] (view) {View Layer\\(Blade Templates)};
    
    % Application Layer
    \node[box, below=of view, fill=orange!20] (controller) {Controller Layer\\(HTTP Controllers)};
    
    % Business Logic Layer
    \node[box, below=of controller, fill=red!20] (model) {Model Layer\\(Eloquent Models)};
    
    % Data Layer
    \node[box, below=of model, fill=purple!20] (db) {Database\\(SQLite/MySQL)};
    
    % Arrows
    \draw[arrow, <->] (browser) -- (view);
    \draw[arrow, <->] (view) -- (controller);
    \draw[arrow, <->] (controller) -- (model);
    \draw[arrow, <->] (model) -- (db);
    
    % Labels
    \node[right=2cm of browser] {\textbf{Client Layer}};
    \node[right=2cm of view] {\textbf{Presentation Layer}};
    \node[right=2cm of controller] {\textbf{Application Layer}};
    \node[right=2cm of model] {\textbf{Business Logic Layer}};
    \node[right=2cm of db] {\textbf{Data Layer}};
    
\end{tikzpicture}
\caption{Arsitektur Sistem SISRI-BPI}
\label{fig:arsitektur-sistem}
\end{figure}

\subsection{Use Case Diagram}
Use case diagram menggambarkan interaksi antara aktor dengan sistem. Diagram use case sistem SISRI-BPI ditunjukkan pada Gambar \ref{fig:usecase}.

\begin{figure}[H]
\centering
\begin{tikzpicture}[
    actor/.style={},
    usecase/.style={ellipse, draw, minimum width=2.5cm, minimum height=1cm, align=center, font=\small},
    system/.style={rectangle, draw, dashed, minimum width=8cm, minimum height=12cm}
]
    % System boundary
    \node[system, label=above:{\textbf{SISRI-BPI}}] (system) at (5,0) {};
    
    % Actors
    \node at (-1, 4) (mahasiswa) {};
    \draw (-1, 4.5) circle (0.3);
    \draw (-1, 4.2) -- (-1, 3.5);
    \draw (-1, 4) -- (-0.5, 3.8);
    \draw (-1, 4) -- (-1.5, 3.8);
    \draw (-1, 3.5) -- (-0.5, 3);
    \draw (-1, 3.5) -- (-1.5, 3);
    \node at (-1, 2.7) {\small Mahasiswa};
    
    \node at (-1, -1) (dosen) {};
    \draw (-1, -0.5) circle (0.3);
    \draw (-1, -0.8) -- (-1, -1.5);
    \draw (-1, -1) -- (-0.5, -1.2);
    \draw (-1, -1) -- (-1.5, -1.2);
    \draw (-1, -1.5) -- (-0.5, -2);
    \draw (-1, -1.5) -- (-1.5, -2);
    \node at (-1, -2.3) {\small Dosen};
    
    \node at (11, 2) (koordinator) {};
    \draw (11, 2.5) circle (0.3);
    \draw (11, 2.2) -- (11, 1.5);
    \draw (11, 2) -- (11.5, 1.8);
    \draw (11, 2) -- (10.5, 1.8);
    \draw (11, 1.5) -- (11.5, 1);
    \draw (11, 1.5) -- (10.5, 1);
    \node at (11, 0.7) {\small Koordinator};
    
    \node at (11, -3) (admin) {};
    \draw (11, -2.5) circle (0.3);
    \draw (11, -2.8) -- (11, -3.5);
    \draw (11, -3) -- (11.5, -3.2);
    \draw (11, -3) -- (10.5, -3.2);
    \draw (11, -3.5) -- (11.5, -4);
    \draw (11, -3.5) -- (10.5, -4);
    \node at (11, -4.3) {\small Admin};
    
    % Use cases
    \node[usecase] (uc1) at (5, 5) {Login};
    \node[usecase] (uc2) at (3, 3.5) {Ajukan Topik};
    \node[usecase] (uc3) at (5, 2) {Bimbingan};
    \node[usecase] (uc4) at (3, 0.5) {Daftar Sidang};
    \node[usecase] (uc5) at (7, 3.5) {Setujui Topik};
    \node[usecase] (uc6) at (7, 1) {Jadwalkan\\Sidang};
    \node[usecase] (uc7) at (5, -1) {Nilai Sidang};
    \node[usecase] (uc8) at (3, -2.5) {Revisi};
    \node[usecase] (uc9) at (7, -2.5) {Kelola Data\\Master};
    \node[usecase] (uc10) at (5, -4.5) {Lihat Laporan};
    
    % Connections - Mahasiswa
    \draw (0, 4) -- (uc1);
    \draw (0, 4) -- (uc2);
    \draw (0, 4) -- (uc3);
    \draw (0, 4) -- (uc4);
    \draw (0, 3.5) -- (uc8);
    
    % Connections - Dosen
    \draw (0, -1) -- (uc1);
    \draw (0, -1) -- (uc3);
    \draw (0, -1) -- (uc7);
    \draw (0, -1) -- (uc8);
    
    % Connections - Koordinator
    \draw (10, 2) -- (uc1);
    \draw (10, 2) -- (uc5);
    \draw (10, 2) -- (uc6);
    \draw (10, 2) -- (uc10);
    
    % Connections - Admin
    \draw (10, -3) -- (uc1);
    \draw (10, -3) -- (uc9);
    \draw (10, -3) -- (uc10);
    
\end{tikzpicture}
\caption{Use Case Diagram SISRI-BPI}
\label{fig:usecase}
\end{figure}

\subsection{Activity Diagram}

\subsubsection{Activity Diagram Pengajuan Topik Skripsi}
\begin{figure}[H]
\centering
\begin{tikzpicture}[
    node distance=1cm,
    startstop/.style={circle, draw, fill=black, minimum size=0.5cm},
    process/.style={rectangle, draw, rounded corners, minimum width=3cm, minimum height=0.8cm, align=center, fill=blue!10},
    decision/.style={diamond, draw, aspect=2, minimum width=1.5cm, fill=yellow!20},
    arrow/.style={->, thick}
]
    % Start
    \node[startstop] (start) {};
    
    % Processes
    \node[process, below=of start] (p1) {Mahasiswa login};
    \node[process, below=of p1] (p2) {Pilih menu Topik Skripsi};
    \node[process, below=of p2] (p3) {Isi form pengajuan topik};
    \node[process, below=of p3] (p4) {Upload file proposal};
    \node[process, below=of p4] (p5) {Submit pengajuan};
    \node[decision, below=of p5] (d1) {Valid?};
    \node[process, below left=1cm and 1cm of d1] (p6) {Tampilkan error};
    \node[process, below right=1cm and 1cm of d1] (p7) {Simpan ke database};
    \node[process, below=2cm of d1] (p8) {Notifikasi ke Koordinator};
    
    % End
    \node[startstop, below=of p8] (end) {};
    \draw[fill=white] (end) circle (0.2cm);
    
    % Arrows
    \draw[arrow] (start) -- (p1);
    \draw[arrow] (p1) -- (p2);
    \draw[arrow] (p2) -- (p3);
    \draw[arrow] (p3) -- (p4);
    \draw[arrow] (p4) -- (p5);
    \draw[arrow] (p5) -- (d1);
    \draw[arrow] (d1) -- node[above] {Tidak} (p6);
    \draw[arrow] (d1) -- node[above] {Ya} (p7);
    \draw[arrow] (p6) |- (p3);
    \draw[arrow] (p7) |- (p8);
    \draw[arrow] (p8) -- (end);
    
\end{tikzpicture}
\caption{Activity Diagram Pengajuan Topik Skripsi}
\label{fig:activity-topik}
\end{figure}

\subsubsection{Activity Diagram Penjadwalan Sidang Otomatis}
\begin{figure}[H]
\centering
\begin{tikzpicture}[
    node distance=0.8cm,
    startstop/.style={circle, draw, fill=black, minimum size=0.5cm},
    process/.style={rectangle, draw, rounded corners, minimum width=3.5cm, minimum height=0.7cm, align=center, fill=blue!10, font=\small},
    decision/.style={diamond, draw, aspect=2, minimum width=1cm, fill=yellow!20, font=\small},
    arrow/.style={->, thick}
]
    % Start
    \node[startstop] (start) {};
    
    % Processes
    \node[process, below=of start] (p1) {Koordinator login};
    \node[process, below=of p1] (p2) {Pilih menu Jadwal Otomatis};
    \node[process, below=of p2] (p3) {Ambil pendaftaran\\status=menunggu};
    \node[process, below=of p3] (p4) {Ambil data ruangan\\dan slot waktu};
    \node[process, below=of p4] (p5) {Jalankan algoritma\\penjadwalan};
    \node[decision, below=of p5] (d1) {Konflik?};
    \node[process, right=2cm of d1] (p6) {Cari slot alternatif};
    \node[process, below=of d1] (p7) {Generate jadwal};
    \node[process, below=of p7] (p8) {Simpan pelaksanaan sidang};
    \node[process, below=of p8] (p9) {Kirim notifikasi};
    
    % End
    \node[startstop, below=of p9] (end) {};
    \draw[fill=white] (end) circle (0.2cm);
    
    % Arrows
    \draw[arrow] (start) -- (p1);
    \draw[arrow] (p1) -- (p2);
    \draw[arrow] (p2) -- (p3);
    \draw[arrow] (p3) -- (p4);
    \draw[arrow] (p4) -- (p5);
    \draw[arrow] (p5) -- (d1);
    \draw[arrow] (d1) -- node[above] {Ya} (p6);
    \draw[arrow] (p6) |- (p5);
    \draw[arrow] (d1) -- node[right] {Tidak} (p7);
    \draw[arrow] (p7) -- (p8);
    \draw[arrow] (p8) -- (p9);
    \draw[arrow] (p9) -- (end);
    
\end{tikzpicture}
\caption{Activity Diagram Penjadwalan Sidang Otomatis}
\label{fig:activity-jadwal}
\end{figure}

\subsection{Sequence Diagram}

\subsubsection{Sequence Diagram Proses Bimbingan}
\begin{figure}[H]
\centering
\begin{sequencediagram}
    \newthread{mhs}{Mahasiswa}
    \newinst[2]{ctrl}{Controller}
    \newinst[2]{model}{Model}
    \newinst[2]{db}{Database}
    
    \begin{call}{mhs}{submitBimbingan()}{ctrl}{response}
        \begin{call}{ctrl}{validate()}{ctrl}{valid}
        \end{call}
        \begin{call}{ctrl}{create()}{model}{bimbingan}
            \begin{call}{model}{insert()}{db}{success}
            \end{call}
        \end{call}
        \begin{call}{ctrl}{notify()}{model}{sent}
        \end{call}
    \end{call}
    
    \begin{sdblock}{alt}{Dosen Merespon}
        \begin{call}{mhs}{getStatus()}{ctrl}{status}
            \begin{call}{ctrl}{find()}{model}{data}
                \begin{call}{model}{select()}{db}{result}
                \end{call}
            \end{call}
        \end{call}
    \end{sdblock}
\end{sequencediagram}
\caption{Sequence Diagram Proses Bimbingan}
\label{fig:sequence-bimbingan}
\end{figure}

\subsection{Class Diagram}
Class diagram menggambarkan struktur kelas-kelas yang digunakan dalam sistem beserta relasinya.

\begin{figure}[H]
\centering
\begin{tikzpicture}[
    class/.style={rectangle, draw, minimum width=4cm, font=\small},
    classlabel/.style={rectangle, draw, fill=blue!20, minimum width=4cm, font=\small\bfseries}
]
    % User Class
    \node[classlabel] (user-label) at (0, 6) {User};
    \node[class, below=-0.02cm of user-label, text width=3.8cm, align=left] (user) {
        - id: int\\
        - name: string\\
        - email: string\\
        - password: string\\
        - role: enum\\
        + login()\\
        + logout()
    };
    
    % Mahasiswa Class
    \node[classlabel] (mhs-label) at (-5, 2) {Mahasiswa};
    \node[class, below=-0.02cm of mhs-label, text width=3.8cm, align=left] (mhs) {
        - nim: string\\
        - nama: string\\
        - angkatan: string\\
        - prodi\_id: int\\
        + ajukanTopik()\\
        + bimbingan()\\
        + daftarSidang()
    };
    
    % Dosen Class
    \node[classlabel] (dosen-label) at (5, 2) {Dosen};
    \node[class, below=-0.02cm of dosen-label, text width=3.8cm, align=left] (dosen) {
        - nip: string\\
        - nama: string\\
        - prodi\_id: int\\
        + setujuiPembimbing()\\
        + responBimbingan()\\
        + nilaiSidang()
    };
    
    % TopikSkripsi Class
    \node[classlabel] (topik-label) at (-5, -3) {TopikSkripsi};
    \node[class, below=-0.02cm of topik-label, text width=3.8cm, align=left] (topik) {
        - judul: string\\
        - status: enum\\
        - bidang\_minat\_id: int\\
        + submit()\\
        + approve()\\
        + reject()
    };
    
    % Bimbingan Class
    \node[classlabel] (bimbingan-label) at (0, -3) {Bimbingan};
    \node[class, below=-0.02cm of bimbingan-label, text width=3.8cm, align=left] (bimbingan) {
        - pokok\_bimbingan: text\\
        - status: enum\\
        - tanggal: datetime\\
        + submit()\\
        + revisi()\\
        + setujui()
    };
    
    % PelaksanaanSidang Class
    \node[classlabel] (sidang-label) at (5, -3) {PelaksanaanSidang};
    \node[class, below=-0.02cm of sidang-label, text width=3.8cm, align=left] (sidang) {
        - tanggal\_sidang: datetime\\
        - tempat: string\\
        - status: enum\\
        + jadwalkan()\\
        + selesai()\\
        + batalkan()
    };
    
    % Relationships
    \draw[->, thick] (user.south) -- ++(0,-0.5) -| node[near start, above] {1} node[near end, above] {1} (mhs-label.north);
    \draw[->, thick] (user.south) -- ++(0,-0.5) -| node[near start, above] {} node[near end, above] {1} (dosen-label.north);
    \draw[->, thick] (mhs.south) -- node[left] {1} node[right] {*} (topik-label.north);
    \draw[->, thick] (topik.east) -- node[above] {1} node[below] {*} (bimbingan-label.west);
    \draw[->, thick] (dosen.south) -- ++(0,-0.5) -| node[near start, left] {*} (bimbingan.north);
    \draw[->, thick] (topik.south) -- ++(0,-1) -| node[near start, left] {1} node[near end, above] {*} (sidang.south);
    
\end{tikzpicture}
\caption{Class Diagram SISRI-BPI (Partial)}
\label{fig:class-diagram}
\end{figure}

\subsection{Perancangan Database}

\subsubsection{Entity Relationship Diagram (ERD)}
Perancangan database sistem SISRI-BPI menggunakan pendekatan relasional dengan Entity Relationship Diagram (ERD) sebagai berikut:

\begin{figure}[H]
\centering
\begin{tikzpicture}[
    entity/.style={rectangle, draw, minimum width=2.5cm, minimum height=0.8cm, fill=blue!20},
    attribute/.style={ellipse, draw, minimum width=1.5cm, font=\tiny},
    relation/.style={diamond, draw, aspect=2, fill=yellow!20, font=\small},
    node distance=1cm
]
    % Entities
    \node[entity] (user) at (0, 4) {Users};
    \node[entity] (mahasiswa) at (-4, 2) {Mahasiswa};
    \node[entity] (dosen) at (4, 2) {Dosen};
    \node[entity] (topik) at (-4, -1) {TopikSkripsi};
    \node[entity] (bimbingan) at (0, -1) {Bimbingan};
    \node[entity] (pendaftaran) at (-4, -4) {PendaftaranSidang};
    \node[entity] (pelaksanaan) at (0, -4) {PelaksanaanSidang};
    \node[entity] (nilai) at (4, -4) {Nilai};
    \node[entity] (penguji) at (4, -1) {PengujiSidang};
    
    % Relations
    \node[relation] (r1) at (-2, 3) {has};
    \node[relation] (r2) at (2, 3) {has};
    \node[relation] (r3) at (-4, 0.5) {owns};
    \node[relation] (r4) at (-2, -1) {has};
    \node[relation] (r5) at (2, -1) {gives};
    \node[relation] (r6) at (-4, -2.5) {creates};
    \node[relation] (r7) at (-2, -4) {generates};
    \node[relation] (r8) at (2, -4) {has};
    
    % Connections
    \draw (user) -- (r1) -- (mahasiswa);
    \draw (user) -- (r2) -- (dosen);
    \draw (mahasiswa) -- (r3) -- (topik);
    \draw (topik) -- (r4) -- (bimbingan);
    \draw (dosen) -- (r5) -- (bimbingan);
    \draw (topik) -- (r6) -- (pendaftaran);
    \draw (pendaftaran) -- (r7) -- (pelaksanaan);
    \draw (pelaksanaan) -- (r8) -- (nilai);
    \draw (dosen) -- (penguji);
    \draw (penguji) -- (pelaksanaan);
    
\end{tikzpicture}
\caption{Entity Relationship Diagram SISRI-BPI}
\label{fig:erd}
\end{figure}

\subsubsection{Skema Database}
Berikut adalah skema tabel-tabel utama dalam database SISRI-BPI:

\begin{table}[H]
\centering
\caption{Struktur Tabel Users}
\label{tab:tabel-users}
\begin{tabular}{|l|l|l|l|}
\hline
\textbf{Field} & \textbf{Type} & \textbf{Key} & \textbf{Keterangan} \\
\hline
id & BIGINT & PK & Primary key, auto increment \\
\hline
name & VARCHAR(255) & - & Nama lengkap pengguna \\
\hline
username & VARCHAR(255) & UNIQUE & Username untuk login \\
\hline
email & VARCHAR(255) & UNIQUE & Email pengguna \\
\hline
password & VARCHAR(255) & - & Password terenkripsi \\
\hline
role & ENUM & - & admin, mahasiswa, dosen, koordinator \\
\hline
is\_active & BOOLEAN & - & Status aktif pengguna \\
\hline
created\_at & TIMESTAMP & - & Waktu pembuatan \\
\hline
updated\_at & TIMESTAMP & - & Waktu update terakhir \\
\hline
\end{tabular}
\end{table}

\begin{table}[H]
\centering
\caption{Struktur Tabel Topik Skripsi}
\label{tab:tabel-topik}
\begin{tabular}{|l|l|l|l|}
\hline
\textbf{Field} & \textbf{Type} & \textbf{Key} & \textbf{Keterangan} \\
\hline
id & BIGINT & PK & Primary key \\
\hline
mahasiswa\_id & BIGINT & FK & Foreign key ke mahasiswa \\
\hline
bidang\_minat\_id & BIGINT & FK & Foreign key ke bidang minat \\
\hline
judul & VARCHAR(500) & - & Judul skripsi \\
\hline
file\_proposal & VARCHAR(255) & - & Path file proposal \\
\hline
status & ENUM & - & menunggu, diterima, ditolak \\
\hline
catatan & TEXT & - & Catatan dari koordinator \\
\hline
created\_at & TIMESTAMP & - & Waktu pengajuan \\
\hline
\end{tabular}
\end{table}

\begin{table}[H]
\centering
\caption{Struktur Tabel Bimbingan}
\label{tab:tabel-bimbingan}
\begin{tabular}{|l|l|l|l|}
\hline
\textbf{Field} & \textbf{Type} & \textbf{Key} & \textbf{Keterangan} \\
\hline
id & BIGINT & PK & Primary key \\
\hline
topik\_id & BIGINT & FK & Foreign key ke topik \\
\hline
dosen\_id & BIGINT & FK & Foreign key ke dosen \\
\hline
jenis & ENUM & - & proposal, skripsi \\
\hline
pokok\_bimbingan & TEXT & - & Materi yang dibahas \\
\hline
file\_bimbingan & VARCHAR(255) & - & File lampiran \\
\hline
pesan\_mahasiswa & TEXT & - & Pesan dari mahasiswa \\
\hline
pesan\_dosen & TEXT & - & Respon dari dosen \\
\hline
status & ENUM & - & menunggu, direvisi, disetujui \\
\hline
tanggal\_bimbingan & TIMESTAMP & - & Tanggal bimbingan \\
\hline
tanggal\_respon & TIMESTAMP & - & Tanggal respon dosen \\
\hline
\end{tabular}
\end{table}

\begin{table}[H]
\centering
\caption{Struktur Tabel Pelaksanaan Sidang}
\label{tab:tabel-pelaksanaan}
\begin{tabular}{|l|l|l|l|}
\hline
\textbf{Field} & \textbf{Type} & \textbf{Key} & \textbf{Keterangan} \\
\hline
id & BIGINT & PK & Primary key \\
\hline
pendaftaran\_sidang\_id & BIGINT & FK & Foreign key ke pendaftaran \\
\hline
tanggal\_sidang & DATETIME & - & Tanggal dan waktu sidang \\
\hline
tempat & VARCHAR(100) & - & Lokasi sidang \\
\hline
status & ENUM & - & dijadwalkan, selesai, dibatalkan \\
\hline
berita\_acara & VARCHAR(255) & - & File berita acara \\
\hline
\end{tabular}
\end{table}

\subsection{Perancangan Antarmuka (Wireframe)}

\subsubsection{Halaman Login}
\begin{figure}[H]
\centering
\begin{tikzpicture}
    % Browser frame
    \draw[rounded corners] (0,0) rectangle (12, 8);
    \draw (0, 7.5) -- (12, 7.5);
    \fill[gray!30] (0.5, 7.6) circle (0.15);
    \fill[gray!30] (1, 7.6) circle (0.15);
    \fill[gray!30] (1.5, 7.6) circle (0.15);
    
    % Left panel
    \fill[blue!10] (0.2, 0.2) rectangle (5.8, 7.3);
    \node at (3, 6) {\textbf{SISRI-BPI}};
    \node at (3, 5.5) {\small Sistem Informasi Skripsi};
    \draw (2, 2) rectangle (4, 4.5);
    \node at (3, 3.25) {\small [Ilustrasi]};
    
    % Right panel - Login form
    \node at (9, 6.5) {\textbf{Selamat Datang}};
    
    % Email input
    \node[anchor=west] at (6.5, 5.5) {\small Email};
    \draw[rounded corners] (6.5, 4.8) rectangle (11.5, 5.3);
    
    % Password input
    \node[anchor=west] at (6.5, 4.3) {\small Password};
    \draw[rounded corners] (6.5, 3.6) rectangle (11.5, 4.1);
    
    % Remember me & Forgot
    \draw (6.5, 3.2) rectangle (6.8, 3);
    \node[anchor=west] at (6.9, 3.1) {\tiny Ingat saya};
    \node[anchor=east] at (11.5, 3.1) {\tiny \color{blue}Lupa Password?};
    
    % Login button
    \fill[blue!60, rounded corners] (6.5, 1.8) rectangle (11.5, 2.5);
    \node[white] at (9, 2.15) {\small Masuk};
    
\end{tikzpicture}
\caption{Wireframe Halaman Login}
\label{fig:wireframe-login}
\end{figure}

\subsubsection{Halaman Dashboard Mahasiswa}
\begin{figure}[H]
\centering
\begin{tikzpicture}
    % Browser frame
    \draw[rounded corners] (0,0) rectangle (14, 9);
    \draw (0, 8.5) -- (14, 8.5);
    
    % Sidebar
    \fill[blue!20] (0.1, 0.1) rectangle (3, 8.4);
    \node at (1.5, 8) {\small \textbf{SISRI}};
    \draw (0.3, 7.3) rectangle (2.8, 7.7);
    \node at (1.55, 7.5) {\tiny Dashboard};
    \draw (0.3, 6.8) rectangle (2.8, 7.2);
    \node at (1.55, 7) {\tiny Topik Skripsi};
    \draw (0.3, 6.3) rectangle (2.8, 6.7);
    \node at (1.55, 6.5) {\tiny Bimbingan};
    \draw (0.3, 5.8) rectangle (2.8, 6.2);
    \node at (1.55, 6) {\tiny Sidang};
    
    % Header
    \fill[white] (3.1, 8) rectangle (13.9, 8.4);
    \node[anchor=west] at (3.3, 8.2) {\small Dashboard};
    \node[anchor=east] at (13.7, 8.2) {\tiny Budi Santoso};
    
    % Stats cards
    \fill[green!20, rounded corners] (3.3, 6.5) rectangle (6, 7.8);
    \node at (4.65, 7.4) {\tiny Bimbingan};
    \node at (4.65, 7) {\small \textbf{5}};
    
    \fill[blue!20, rounded corners] (6.2, 6.5) rectangle (8.9, 7.8);
    \node at (7.55, 7.4) {\tiny Topik};
    \node at (7.55, 7) {\small \textbf{Diterima}};
    
    \fill[yellow!20, rounded corners] (9.1, 6.5) rectangle (11.8, 7.8);
    \node at (10.45, 7.4) {\tiny Sidang};
    \node at (10.45, 7) {\small \textbf{Dijadwalkan}};
    
    % Recent activity
    \draw[rounded corners] (3.3, 0.3) rectangle (13.7, 6.2);
    \node[anchor=west] at (3.5, 5.9) {\small \textbf{Aktivitas Terbaru}};
    \draw (3.3, 5.5) -- (13.7, 5.5);
    
    \node[anchor=west] at (3.5, 5) {\tiny Bimbingan BAB 3 - Menunggu respon};
    \node[anchor=west] at (3.5, 4.5) {\tiny Sidang Proposal - 10 Des 2025};
    \node[anchor=west] at (3.5, 4) {\tiny Revisi BAB 2 - Disetujui};
    
\end{tikzpicture}
\caption{Wireframe Dashboard Mahasiswa}
\label{fig:wireframe-dashboard}
\end{figure}

% ============================================
\section{Flowchart Sistem}

\subsection{Flowchart Alur Utama Skripsi}
\begin{figure}[H]
\centering
\begin{tikzpicture}[
    node distance=1cm,
    startstop/.style={ellipse, draw, minimum width=2cm, fill=green!20},
    process/.style={rectangle, draw, minimum width=3cm, minimum height=0.8cm, fill=blue!10},
    decision/.style={diamond, draw, aspect=2, fill=yellow!20},
    io/.style={trapezium, draw, trapezium left angle=70, trapezium right angle=110, minimum width=2cm, fill=orange!20},
    arrow/.style={->, thick}
]
    % Start
    \node[startstop] (start) {Mulai};
    
    % Main flow
    \node[process, below=of start] (p1) {Ajukan Topik Skripsi};
    \node[decision, below=of p1] (d1) {Disetujui?};
    \node[process, below=of d1] (p2) {Proses Bimbingan};
    \node[decision, below=of p2] (d2) {Min. 8x?};
    \node[process, below=of d2] (p3) {Daftar Sempro};
    \node[decision, below=of p3] (d3) {Lulus?};
    \node[process, right=2cm of d3] (p4) {Revisi Sempro};
    \node[process, below=of d3] (p5) {Lanjut Bimbingan};
    \node[decision, below=of p5] (d4) {Min. 8x?};
    \node[process, below=of d4] (p6) {Daftar Sidang};
    \node[decision, below=of p6] (d5) {Lulus?};
    \node[process, right=2cm of d5] (p7) {Revisi Skripsi};
    \node[startstop, below=of d5] (end) {Selesai};
    
    % Arrows
    \draw[arrow] (start) -- (p1);
    \draw[arrow] (p1) -- (d1);
    \draw[arrow] (d1) -- node[right] {Ya} (p2);
    \draw[arrow] (d1.west) -- ++(-1.5,0) |- node[near start, above] {Tidak} (p1);
    \draw[arrow] (p2) -- (d2);
    \draw[arrow] (d2) -- node[right] {Ya} (p3);
    \draw[arrow] (d2.west) -- ++(-1,0) |- node[near start, above] {Tidak} (p2);
    \draw[arrow] (p3) -- (d3);
    \draw[arrow] (d3) -- node[above] {Tidak} (p4);
    \draw[arrow] (p4) |- (p3);
    \draw[arrow] (d3) -- node[right] {Ya} (p5);
    \draw[arrow] (p5) -- (d4);
    \draw[arrow] (d4) -- node[right] {Ya} (p6);
    \draw[arrow] (d4.west) -- ++(-1,0) |- node[near start, above] {Tidak} (p5);
    \draw[arrow] (p6) -- (d5);
    \draw[arrow] (d5) -- node[above] {Tidak} (p7);
    \draw[arrow] (p7) |- (p6);
    \draw[arrow] (d5) -- node[right] {Ya} (end);
    
\end{tikzpicture}
\caption{Flowchart Alur Utama Proses Skripsi}
\label{fig:flowchart-main}
\end{figure}


% Include BAB 5 - Implementasi
% ============================================
% BAB 5 - IMPLEMENTASI SISTEM
% SISRI-BPI (Sistem Informasi Skripsi)
% Universitas Trunojoyo Madura
% ============================================

\chapter{IMPLEMENTASI SISTEM}

\section{Lingkungan Implementasi}

\subsection{Spesifikasi Perangkat Keras}
Implementasi sistem SISRI-BPI dilakukan pada perangkat dengan spesifikasi sebagai berikut:

\begin{table}[H]
\centering
\caption{Spesifikasi Perangkat Keras}
\label{tab:spesifikasi-hardware}
\begin{tabular}{|l|l|}
\hline
\textbf{Komponen} & \textbf{Spesifikasi} \\
\hline
Processor & Intel Core i5 / AMD Ryzen 5 atau lebih tinggi \\
\hline
RAM & Minimal 8 GB \\
\hline
Storage & SSD 256 GB atau lebih \\
\hline
Display & Resolusi minimal 1920 x 1080 \\
\hline
\end{tabular}
\end{table}

\subsection{Spesifikasi Perangkat Lunak}
\begin{table}[H]
\centering
\caption{Spesifikasi Perangkat Lunak}
\label{tab:spesifikasi-software}
\begin{tabular}{|l|l|l|}
\hline
\textbf{Software} & \textbf{Versi} & \textbf{Keterangan} \\
\hline
Operating System & Linux/Windows/macOS & Development \& Production \\
\hline
PHP & 8.4.x & Backend Runtime \\
\hline
Laravel Framework & 12.x & PHP Framework \\
\hline
Node.js & 20.x LTS & Frontend Build Tool \\
\hline
SQLite/MySQL & 3.x / 8.x & Database Management \\
\hline
Composer & 2.x & PHP Dependency Manager \\
\hline
NPM & 10.x & Node Package Manager \\
\hline
Visual Studio Code & Latest & Code Editor \\
\hline
Git & Latest & Version Control \\
\hline
\end{tabular}
\end{table}

\subsection{Teknologi yang Digunakan}
\begin{table}[H]
\centering
\caption{Stack Teknologi SISRI-BPI}
\label{tab:tech-stack}
\begin{tabular}{|l|l|p{7cm}|}
\hline
\textbf{Layer} & \textbf{Teknologi} & \textbf{Fungsi} \\
\hline
Frontend & TailwindCSS & Utility-first CSS framework untuk styling \\
\hline
Frontend & Alpine.js & Lightweight JavaScript framework untuk interaktivitas \\
\hline
Frontend & SweetAlert2 & Library untuk dialog dan notifikasi \\
\hline
Backend & Laravel 12 & PHP framework dengan arsitektur MVC \\
\hline
Backend & Laravel Breeze & Authentication scaffolding \\
\hline
Backend & Spatie Permission & Role \& permission management \\
\hline
Database & SQLite/MySQL & Relational database management system \\
\hline
Build Tool & Vite & Frontend asset bundler \\
\hline
Server & Laravel Octane & High-performance application server \\
\hline
\end{tabular}
\end{table}

% ============================================
\section{Struktur Direktori Proyek}

Struktur direktori proyek SISRI-BPI mengikuti konvensi Laravel dengan beberapa penyesuaian:

\begin{lstlisting}[language=bash, caption={Struktur Direktori SISRI-BPI}, label={lst:struktur-direktori}]
SISRI-BPI/
├── app/
│   ├── Http/
│   │   ├── Controllers/
│   │   │   ├── Admin/
│   │   │   ├── Dosen/
│   │   │   ├── Koordinator/
│   │   │   └── Mahasiswa/
│   │   ├── Middleware/
│   │   └── Requests/
│   ├── Models/
│   │   ├── User.php
│   │   ├── Mahasiswa.php
│   │   ├── Dosen.php
│   │   ├── TopikSkripsi.php
│   │   ├── Bimbingan.php
│   │   ├── PendaftaranSidang.php
│   │   ├── PelaksanaanSidang.php
│   │   └── ...
│   └── Providers/
├── database/
│   ├── migrations/
│   └── seeders/
├── resources/
│   ├── views/
│   │   ├── admin/
│   │   ├── dosen/
│   │   ├── koordinator/
│   │   ├── mahasiswa/
│   │   ├── layouts/
│   │   └── components/
│   ├── css/
│   └── js/
├── routes/
│   ├── web.php
│   └── auth.php
└── public/
\end{lstlisting}

% ============================================
\section{Implementasi Database}

\subsection{Implementasi Migration}
Laravel menggunakan migration untuk membuat dan mengelola struktur database. Berikut contoh implementasi migration untuk tabel \texttt{topik\_skripsi}:

\begin{lstlisting}[language=PHP, caption={Migration Tabel Topik Skripsi}, label={lst:migration-topik}]
<?php

use Illuminate\Database\Migrations\Migration;
use Illuminate\Database\Schema\Blueprint;
use Illuminate\Support\Facades\Schema;

return new class extends Migration
{
    public function up(): void
    {
        Schema::create('topik_skripsi', function (Blueprint $table) {
            $table->id();
            $table->foreignId('mahasiswa_id')
                  ->constrained('mahasiswa')
                  ->cascadeOnDelete();
            $table->foreignId('bidang_minat_id')
                  ->constrained('bidang_minat')
                  ->cascadeOnDelete();
            $table->string('judul', 500);
            $table->string('file_proposal', 255)->nullable();
            $table->enum('status', ['menunggu', 'diterima', 'ditolak'])
                  ->default('menunggu');
            $table->text('catatan')->nullable();
            $table->timestamps();
        });
    }

    public function down(): void
    {
        Schema::dropIfExists('topik_skripsi');
    }
};
\end{lstlisting}

\subsection{Implementasi Model}
Model Eloquent digunakan untuk berinteraksi dengan database. Berikut implementasi model \texttt{TopikSkripsi}:

\begin{lstlisting}[language=PHP, caption={Model TopikSkripsi}, label={lst:model-topik}]
<?php

namespace App\Models;

use Illuminate\Database\Eloquent\Model;
use Illuminate\Database\Eloquent\Relations\BelongsTo;
use Illuminate\Database\Eloquent\Relations\HasMany;

class TopikSkripsi extends Model
{
    protected $table = 'topik_skripsi';

    protected $fillable = [
        'mahasiswa_id',
        'bidang_minat_id',
        'judul',
        'file_proposal',
        'status',
        'catatan',
    ];

    public function mahasiswa(): BelongsTo
    {
        return $this->belongsTo(Mahasiswa::class);
    }

    public function bidangMinat(): BelongsTo
    {
        return $this->belongsTo(BidangMinat::class);
    }

    public function usulanPembimbing(): HasMany
    {
        return $this->hasMany(UsulanPembimbing::class, 'topik_id');
    }

    public function bimbingan(): HasMany
    {
        return $this->hasMany(Bimbingan::class, 'topik_id');
    }

    public function pendaftaranSidang(): HasMany
    {
        return $this->hasMany(PendaftaranSidang::class, 'topik_id');
    }

    // Accessor untuk mendapatkan pembimbing yang sudah diterima
    public function getPembimbing1Attribute()
    {
        return $this->usulanPembimbing()
            ->where('urutan', 1)
            ->where('status', 'diterima')
            ->first()?->dosen;
    }

    public function getPembimbing2Attribute()
    {
        return $this->usulanPembimbing()
            ->where('urutan', 2)
            ->where('status', 'diterima')
            ->first()?->dosen;
    }
}
\end{lstlisting}

% ============================================
\section{Implementasi Controller}

\subsection{Controller Bimbingan Mahasiswa}
Berikut implementasi controller untuk mengelola bimbingan dari sisi mahasiswa:

\begin{lstlisting}[language=PHP, caption={BimbinganController untuk Mahasiswa}, label={lst:controller-bimbingan}]
<?php

namespace App\Http\Controllers\Mahasiswa;

use App\Http\Controllers\Controller;
use App\Models\Bimbingan;
use App\Models\BimbinganHistory;
use Illuminate\Http\Request;
use Illuminate\Support\Facades\Auth;
use Illuminate\Support\Facades\Storage;

class BimbinganController extends Controller
{
    public function index()
    {
        $mahasiswa = Auth::user()->mahasiswa;
        $topik = $mahasiswa->topikSkripsi()
            ->where('status', 'diterima')
            ->first();

        if (!$topik) {
            return redirect()->route('mahasiswa.topik.index')
                ->with('error', 'Anda belum memiliki topik yang disetujui');
        }

        $bimbingan = Bimbingan::where('topik_id', $topik->id)
            ->with(['dosen', 'histories'])
            ->orderBy('created_at', 'desc')
            ->paginate(10);

        return view('mahasiswa.bimbingan.index', compact('bimbingan', 'topik'));
    }

    public function store(Request $request)
    {
        $request->validate([
            'dosen_id' => 'required|exists:dosen,id',
            'jenis' => 'required|in:proposal,skripsi',
            'pokok_bimbingan' => 'required|string|max:1000',
            'pesan_mahasiswa' => 'nullable|string|max:2000',
            'file_bimbingan' => 'nullable|file|mimes:pdf,doc,docx|max:10240',
        ]);

        $mahasiswa = Auth::user()->mahasiswa;
        $topik = $mahasiswa->topikSkripsi()
            ->where('status', 'diterima')
            ->firstOrFail();

        $filePath = null;
        if ($request->hasFile('file_bimbingan')) {
            $filePath = $request->file('file_bimbingan')
                ->store('bimbingan', 'public');
        }

        $bimbingan = Bimbingan::create([
            'topik_id' => $topik->id,
            'dosen_id' => $request->dosen_id,
            'jenis' => $request->jenis,
            'pokok_bimbingan' => $request->pokok_bimbingan,
            'pesan_mahasiswa' => $request->pesan_mahasiswa,
            'file_bimbingan' => $filePath,
            'status' => 'menunggu',
            'tanggal_bimbingan' => now(),
        ]);

        // Catat history
        BimbinganHistory::create([
            'bimbingan_id' => $bimbingan->id,
            'status' => 'menunggu',
            'aksi' => 'submit',
            'catatan' => 'Mahasiswa mengajukan bimbingan',
            'oleh' => $mahasiswa->nama,
        ]);

        return redirect()->route('mahasiswa.bimbingan.index')
            ->with('success', 'Bimbingan berhasil diajukan');
    }
}
\end{lstlisting}

\subsection{Controller Penjadwalan Otomatis}
Implementasi fitur penjadwalan sidang otomatis oleh koordinator:

\begin{lstlisting}[language=PHP, caption={JadwalOtomatisController}, label={lst:controller-jadwal}]
<?php

namespace App\Http\Controllers\Koordinator;

use App\Http\Controllers\Controller;
use App\Models\PendaftaranSidang;
use App\Models\PelaksanaanSidang;
use App\Models\PengujiSidang;
use App\Models\Ruangan;
use Illuminate\Http\Request;
use Carbon\Carbon;

class JadwalOtomatisController extends Controller
{
    public function generate(Request $request)
    {
        $request->validate([
            'jadwal_sidang_id' => 'required|exists:jadwal_sidang,id',
            'tanggal_mulai' => 'required|date',
            'tanggal_selesai' => 'required|date|after_or_equal:tanggal_mulai',
        ]);

        // Ambil pendaftaran yang menunggu
        $pendaftarans = PendaftaranSidang::where('jadwal_sidang_id', $request->jadwal_sidang_id)
            ->where('status_koordinator', 'menunggu')
            ->where('status_pembimbing_1', 'disetujui')
            ->where('status_pembimbing_2', 'disetujui')
            ->with(['topik.usulanPembimbing.dosen'])
            ->get();

        // Ambil ruangan aktif
        $ruangans = Ruangan::where('is_active', true)->get();
        
        // Slot waktu (08:00 - 16:00, @1.5 jam)
        $slots = ['08:00', '09:30', '11:00', '13:00', '14:30'];
        
        $scheduled = 0;
        $currentDate = Carbon::parse($request->tanggal_mulai);
        $endDate = Carbon::parse($request->tanggal_selesai);

        foreach ($pendaftarans as $pendaftaran) {
            // Cari slot kosong
            $slotFound = false;
            
            while ($currentDate <= $endDate && !$slotFound) {
                // Skip weekend
                if ($currentDate->isWeekend()) {
                    $currentDate->addDay();
                    continue;
                }

                foreach ($ruangans as $ruangan) {
                    foreach ($slots as $slot) {
                        $waktu = $currentDate->copy()
                            ->setTimeFromTimeString($slot);

                        // Cek konflik ruangan
                        $conflict = PelaksanaanSidang::where('tanggal_sidang', $waktu)
                            ->where('tempat', 'LIKE', '%' . $ruangan->nama . '%')
                            ->exists();

                        if (!$conflict) {
                            // Buat pelaksanaan sidang
                            $pelaksanaan = PelaksanaanSidang::create([
                                'pendaftaran_sidang_id' => $pendaftaran->id,
                                'tanggal_sidang' => $waktu,
                                'tempat' => $ruangan->nama . ' - ' . $ruangan->lokasi,
                                'status' => 'dijadwalkan',
                            ]);

                            // Set penguji dari pembimbing
                            $this->assignPenguji($pelaksanaan, $pendaftaran);

                            // Update status pendaftaran
                            $pendaftaran->update([
                                'status_koordinator' => 'disetujui'
                            ]);

                            $scheduled++;
                            $slotFound = true;
                            break 2;
                        }
                    }
                }
                $currentDate->addDay();
            }
        }

        return redirect()->back()
            ->with('success', "Berhasil menjadwalkan {$scheduled} sidang");
    }

    private function assignPenguji($pelaksanaan, $pendaftaran)
    {
        $pembimbings = $pendaftaran->topik->usulanPembimbing()
            ->where('status', 'diterima')
            ->orderBy('urutan')
            ->get();

        foreach ($pembimbings as $p) {
            PengujiSidang::create([
                'pelaksanaan_sidang_id' => $pelaksanaan->id,
                'dosen_id' => $p->dosen_id,
                'role' => 'pembimbing_' . $p->urutan,
            ]);
        }
    }
}
\end{lstlisting}

% ============================================
\section{Implementasi View}

\subsection{Layout Utama}
Sistem menggunakan Blade templating engine dengan layout berbasis komponen:

\begin{lstlisting}[language=HTML, caption={Layout Utama Aplikasi}, label={lst:layout-app}]
<!DOCTYPE html>
<html lang="{{ str_replace('_', '-', app()->getLocale()) }}">
<head>
    <meta charset="utf-8">
    <meta name="viewport" content="width=device-width, initial-scale=1">
    <meta name="csrf-token" content="{{ csrf_token() }}">

    <title>{{ config('app.name', 'SISRI-BPI') }}</title>

    <!-- Fonts -->
    <link rel="preconnect" href="https://fonts.bunny.net">
    <link href="https://fonts.bunny.net/css?family=figtree:400,500,600&display=swap" 
          rel="stylesheet" />

    <!-- Scripts -->
    @vite(['resources/css/app.css', 'resources/js/app.js'])
</head>
<body class="font-sans antialiased">
    <div class="min-h-screen bg-gray-100">
        <!-- Sidebar -->
        @include('layouts.sidebar')

        <!-- Page Content -->
        <div class="lg:pl-64">
            <!-- Top Navigation -->
            @include('layouts.navigation')

            <!-- Main Content -->
            <main class="py-6 px-4 sm:px-6 lg:px-8">
                <!-- Flash Messages -->
                @if(session('success'))
                    <x-alert type="success" :message="session('success')" />
                @endif

                @if(session('error'))
                    <x-alert type="error" :message="session('error')" />
                @endif

                {{ $slot }}
            </main>
        </div>
    </div>

    <!-- SweetAlert2 -->
    <script src="https://cdn.jsdelivr.net/npm/sweetalert2@11"></script>
    @stack('scripts')
</body>
</html>
\end{lstlisting}

\subsection{Halaman Daftar Bimbingan}
\begin{lstlisting}[language=HTML, caption={View Daftar Bimbingan Mahasiswa}, label={lst:view-bimbingan}]
<x-app-layout>
    <x-slot name="header">
        <h2 class="text-xl font-semibold text-gray-800">
            Bimbingan Skripsi
        </h2>
    </x-slot>

    <div class="bg-white rounded-lg shadow-md p-6">
        <!-- Header dengan tombol tambah -->
        <div class="flex justify-between items-center mb-6">
            <h3 class="text-lg font-medium">Riwayat Bimbingan</h3>
            <a href="{{ route('mahasiswa.bimbingan.create') }}"
               class="bg-blue-600 hover:bg-blue-700 text-white 
                      px-4 py-2 rounded-lg transition-colors">
                + Ajukan Bimbingan
            </a>
        </div>

        <!-- Tabel Bimbingan -->
        <div class="overflow-x-auto">
            <table class="min-w-full divide-y divide-gray-200">
                <thead class="bg-gray-50">
                    <tr>
                        <th class="px-6 py-3 text-left text-xs font-medium 
                                   text-gray-500 uppercase">No</th>
                        <th class="px-6 py-3 text-left text-xs font-medium 
                                   text-gray-500 uppercase">Tanggal</th>
                        <th class="px-6 py-3 text-left text-xs font-medium 
                                   text-gray-500 uppercase">Pembimbing</th>
                        <th class="px-6 py-3 text-left text-xs font-medium 
                                   text-gray-500 uppercase">Pokok Bahasan</th>
                        <th class="px-6 py-3 text-left text-xs font-medium 
                                   text-gray-500 uppercase">Status</th>
                        <th class="px-6 py-3 text-left text-xs font-medium 
                                   text-gray-500 uppercase">Aksi</th>
                    </tr>
                </thead>
                <tbody class="divide-y divide-gray-200">
                    @forelse($bimbingan as $index => $item)
                    <tr class="hover:bg-gray-50">
                        <td class="px-6 py-4 text-sm">
                            {{ $bimbingan->firstItem() + $index }}
                        </td>
                        <td class="px-6 py-4 text-sm">
                            {{ $item->tanggal_bimbingan->format('d M Y') }}
                        </td>
                        <td class="px-6 py-4 text-sm">
                            {{ $item->dosen->nama }}
                        </td>
                        <td class="px-6 py-4 text-sm">
                            {{ Str::limit($item->pokok_bimbingan, 50) }}
                        </td>
                        <td class="px-6 py-4">
                            <x-status-badge :status="$item->status" />
                        </td>
                        <td class="px-6 py-4 text-sm">
                            <a href="{{ route('mahasiswa.bimbingan.show', $item) }}"
                               class="text-blue-600 hover:text-blue-800">
                                Detail
                            </a>
                        </td>
                    </tr>
                    @empty
                    <tr>
                        <td colspan="6" class="px-6 py-4 text-center text-gray-500">
                            Belum ada data bimbingan
                        </td>
                    </tr>
                    @endforelse
                </tbody>
            </table>
        </div>

        <!-- Pagination -->
        <div class="mt-4">
            {{ $bimbingan->links() }}
        </div>
    </div>
</x-app-layout>
\end{lstlisting}

% ============================================
\section{Implementasi Autentikasi dan Otorisasi}

\subsection{Middleware Role-Based Access Control}
\begin{lstlisting}[language=PHP, caption={Middleware RoleMiddleware}, label={lst:middleware-role}]
<?php

namespace App\Http\Middleware;

use Closure;
use Illuminate\Http\Request;

class RoleMiddleware
{
    public function handle(Request $request, Closure $next, ...$roles)
    {
        if (!$request->user()) {
            return redirect()->route('login');
        }

        if (!$request->user()->hasAnyRole($roles)) {
            abort(403, 'Unauthorized action.');
        }

        return $next($request);
    }
}
\end{lstlisting}

\subsection{Konfigurasi Route dengan Middleware}
\begin{lstlisting}[language=PHP, caption={Konfigurasi Route dengan Role Middleware}, label={lst:route-middleware}]
<?php

use Illuminate\Support\Facades\Route;

// Route Mahasiswa
Route::middleware(['auth', 'role:mahasiswa'])
    ->prefix('mahasiswa')
    ->name('mahasiswa.')
    ->group(function () {
        Route::get('/dashboard', [DashboardController::class, 'index'])
            ->name('dashboard');
        Route::resource('topik', TopikController::class);
        Route::resource('bimbingan', BimbinganController::class);
        Route::resource('sidang', SidangController::class);
    });

// Route Dosen
Route::middleware(['auth', 'role:dosen'])
    ->prefix('dosen')
    ->name('dosen.')
    ->group(function () {
        Route::get('/dashboard', [DosenDashboardController::class, 'index'])
            ->name('dashboard');
        Route::resource('bimbingan', DosenBimbinganController::class);
        Route::resource('nilai', NilaiController::class);
    });

// Route Koordinator
Route::middleware(['auth', 'role:koordinator'])
    ->prefix('koordinator')
    ->name('koordinator.')
    ->group(function () {
        Route::get('/dashboard', [KoordinatorDashboardController::class, 'index'])
            ->name('dashboard');
        Route::resource('topik', KoordinatorTopikController::class);
        Route::post('/jadwal-otomatis', [JadwalOtomatisController::class, 'generate'])
            ->name('jadwal.generate');
    });

// Route Admin
Route::middleware(['auth', 'role:admin'])
    ->prefix('admin')
    ->name('admin.')
    ->group(function () {
        Route::get('/dashboard', [AdminDashboardController::class, 'index'])
            ->name('dashboard');
        Route::resource('users', UserController::class);
        Route::resource('dosen', DosenController::class);
        Route::resource('mahasiswa', MahasiswaController::class);
    });
\end{lstlisting}

% ============================================
\section{Implementasi Fitur Utama}

\subsection{Fitur Upload File}
\begin{lstlisting}[language=PHP, caption={Implementasi Upload File}, label={lst:upload-file}]
<?php

// Di Controller
public function uploadProposal(Request $request)
{
    $request->validate([
        'file_proposal' => 'required|file|mimes:pdf|max:5120', // Max 5MB
    ]);

    if ($request->hasFile('file_proposal')) {
        // Hapus file lama jika ada
        if ($this->topik->file_proposal) {
            Storage::disk('public')->delete($this->topik->file_proposal);
        }

        // Upload file baru
        $file = $request->file('file_proposal');
        $filename = 'proposal_' . $this->topik->mahasiswa->nim . '_' 
                  . time() . '.' . $file->getClientOriginalExtension();
        
        $path = $file->storeAs('proposals', $filename, 'public');

        $this->topik->update(['file_proposal' => $path]);
    }

    return back()->with('success', 'File proposal berhasil diupload');
}
\end{lstlisting}

\subsection{Fitur Notifikasi Real-time}
\begin{lstlisting}[language=JavaScript, caption={Implementasi Notifikasi dengan SweetAlert2}, label={lst:notifikasi}]
// resources/js/notifications.js

// Flash message handler
document.addEventListener('DOMContentLoaded', function() {
    // Success notification
    const successMsg = document.querySelector('[data-success-message]');
    if (successMsg) {
        Swal.fire({
            icon: 'success',
            title: 'Berhasil!',
            text: successMsg.dataset.successMessage,
            timer: 3000,
            showConfirmButton: false
        });
    }

    // Error notification
    const errorMsg = document.querySelector('[data-error-message]');
    if (errorMsg) {
        Swal.fire({
            icon: 'error',
            title: 'Gagal!',
            text: errorMsg.dataset.errorMessage
        });
    }
});

// Confirm delete
function confirmDelete(formId) {
    Swal.fire({
        title: 'Apakah Anda yakin?',
        text: "Data yang dihapus tidak dapat dikembalikan!",
        icon: 'warning',
        showCancelButton: true,
        confirmButtonColor: '#d33',
        cancelButtonColor: '#3085d6',
        confirmButtonText: 'Ya, hapus!',
        cancelButtonText: 'Batal'
    }).then((result) => {
        if (result.isConfirmed) {
            document.getElementById(formId).submit();
        }
    });
}
\end{lstlisting}

% ============================================
\section{Hasil Implementasi}

\subsection{Halaman Login}
Halaman login merupakan halaman pertama yang diakses pengguna untuk masuk ke sistem. Tampilan halaman login dapat dilihat pada Gambar \ref{fig:impl-login}.

\begin{figure}[H]
\centering
\fbox{\includegraphics[width=0.8\textwidth]{images/login.png}}
\caption{Implementasi Halaman Login}
\label{fig:impl-login}
\end{figure}

\subsection{Dashboard Mahasiswa}
Dashboard mahasiswa menampilkan ringkasan informasi terkait proses skripsi mahasiswa yang bersangkutan.

\begin{figure}[H]
\centering
\fbox{\includegraphics[width=0.9\textwidth]{images/dashboard-mahasiswa.png}}
\caption{Implementasi Dashboard Mahasiswa}
\label{fig:impl-dashboard-mhs}
\end{figure}

\subsection{Halaman Bimbingan}
Halaman bimbingan memungkinkan mahasiswa untuk mengajukan dan memantau proses bimbingan dengan dosen pembimbing.

\begin{figure}[H]
\centering
\fbox{\includegraphics[width=0.9\textwidth]{images/bimbingan.png}}
\caption{Implementasi Halaman Bimbingan}
\label{fig:impl-bimbingan}
\end{figure}

\subsection{Halaman Penjadwalan Sidang}
Koordinator dapat melakukan penjadwalan sidang secara otomatis melalui halaman ini.

\begin{figure}[H]
\centering
\fbox{\includegraphics[width=0.9\textwidth]{images/jadwal-sidang.png}}
\caption{Implementasi Halaman Penjadwalan Sidang}
\label{fig:impl-jadwal}
\end{figure}

\subsection{Halaman Input Nilai}
Dosen penguji dapat memasukkan nilai sidang melalui form yang telah disediakan.

\begin{figure}[H]
\centering
\fbox{\includegraphics[width=0.9\textwidth]{images/input-nilai.png}}
\caption{Implementasi Halaman Input Nilai}
\label{fig:impl-nilai}
\end{figure}

% ============================================
\section{Pengujian Sistem}

\subsection{Pengujian Black-Box}
Pengujian black-box dilakukan untuk memastikan fungsionalitas sistem berjalan sesuai dengan kebutuhan.

\begin{table}[H]
\centering
\caption{Hasil Pengujian Black-Box}
\label{tab:pengujian-blackbox}
\begin{tabular}{|c|p{4cm}|p{4cm}|p{3cm}|c|}
\hline
\textbf{No} & \textbf{Skenario Pengujian} & \textbf{Hasil yang Diharapkan} & \textbf{Hasil Pengujian} & \textbf{Status} \\
\hline
1 & Login dengan kredensial valid & Masuk ke dashboard sesuai role & Sesuai harapan & \checkmark \\
\hline
2 & Login dengan kredensial invalid & Menampilkan pesan error & Sesuai harapan & \checkmark \\
\hline
3 & Mahasiswa mengajukan topik & Topik tersimpan dengan status menunggu & Sesuai harapan & \checkmark \\
\hline
4 & Koordinator menyetujui topik & Status topik berubah menjadi diterima & Sesuai harapan & \checkmark \\
\hline
5 & Mahasiswa mengajukan bimbingan & Bimbingan tersimpan dan notifikasi terkirim & Sesuai harapan & \checkmark \\
\hline
6 & Dosen merespon bimbingan & Status bimbingan terupdate & Sesuai harapan & \checkmark \\
\hline
7 & Koordinator generate jadwal otomatis & Jadwal sidang terbuat tanpa konflik & Sesuai harapan & \checkmark \\
\hline
8 & Dosen input nilai sidang & Nilai tersimpan di database & Sesuai harapan & \checkmark \\
\hline
\end{tabular}
\end{table}

\subsection{Pengujian Responsivitas}
Sistem diuji pada berbagai ukuran layar untuk memastikan tampilan responsif.

\begin{table}[H]
\centering
\caption{Hasil Pengujian Responsivitas}
\label{tab:pengujian-responsif}
\begin{tabular}{|l|c|c|c|}
\hline
\textbf{Device} & \textbf{Resolusi} & \textbf{Browser} & \textbf{Status} \\
\hline
Desktop & 1920x1080 & Chrome, Firefox & \checkmark \\
\hline
Laptop & 1366x768 & Chrome, Edge & \checkmark \\
\hline
Tablet & 768x1024 & Safari, Chrome & \checkmark \\
\hline
Mobile & 375x667 & Chrome, Safari & \checkmark \\
\hline
\end{tabular}
\end{table}

\subsection{Pengujian Performa}
Pengujian performa dilakukan untuk mengukur waktu respon sistem.

\begin{table}[H]
\centering
\caption{Hasil Pengujian Performa}
\label{tab:pengujian-performa}
\begin{tabular}{|l|c|c|}
\hline
\textbf{Halaman} & \textbf{Waktu Muat (ms)} & \textbf{Status} \\
\hline
Login & 245 & Baik \\
\hline
Dashboard & 312 & Baik \\
\hline
List Bimbingan & 287 & Baik \\
\hline
Form Pengajuan & 198 & Baik \\
\hline
Generate Jadwal Otomatis & 1245 & Cukup \\
\hline
\end{tabular}
\end{table}


% ============================================
% BACK MATTER
% ============================================

% Daftar Pustaka
\begin{thebibliography}{99}
\addcontentsline{toc}{chapter}{DAFTAR PUSTAKA}

\bibitem{laravel}
Laravel Documentation. (2025). \textit{Laravel - The PHP Framework For Web Artisans}. Retrieved from https://laravel.com/docs

\bibitem{tailwind}
Tailwind CSS. (2025). \textit{Tailwind CSS - Rapidly build modern websites without ever leaving your HTML}. Retrieved from https://tailwindcss.com/docs

\bibitem{pressman}
Pressman, R. S., \& Maxim, B. R. (2020). \textit{Software Engineering: A Practitioner's Approach} (9th ed.). McGraw-Hill Education.

\bibitem{sommerville}
Sommerville, I. (2016). \textit{Software Engineering} (10th ed.). Pearson Education.

\bibitem{spatie}
Spatie. (2025). \textit{Laravel Permission}. Retrieved from https://spatie.be/docs/laravel-permission

\bibitem{alpine}
Alpine.js. (2025). \textit{Alpine.js - Your new, lightweight, JavaScript framework}. Retrieved from https://alpinejs.dev/

\end{thebibliography}

% ============================================
% LAMPIRAN
% ============================================
\appendix
\chapter{LAMPIRAN}

\section{Kode Program Lengkap}
Kode program lengkap dapat diakses melalui repository GitHub:\\
\url{https://github.com/AchmadLutfi196/SISRI-BPI}

\section{Dokumentasi API}
[Dokumentasi API sistem jika ada]

\section{Manual Pengguna}
[Panduan penggunaan sistem untuk setiap role pengguna]

\end{document}
